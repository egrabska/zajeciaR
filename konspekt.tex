\documentclass[]{article}
\usepackage{lmodern}
\usepackage{amssymb,amsmath}
\usepackage{ifxetex,ifluatex}
\usepackage{fixltx2e} % provides \textsubscript
\ifnum 0\ifxetex 1\fi\ifluatex 1\fi=0 % if pdftex
  \usepackage[T1]{fontenc}
  \usepackage[utf8]{inputenc}
\else % if luatex or xelatex
  \ifxetex
    \usepackage{mathspec}
  \else
    \usepackage{fontspec}
  \fi
  \defaultfontfeatures{Ligatures=TeX,Scale=MatchLowercase}
\fi
% use upquote if available, for straight quotes in verbatim environments
\IfFileExists{upquote.sty}{\usepackage{upquote}}{}
% use microtype if available
\IfFileExists{microtype.sty}{%
\usepackage{microtype}
\UseMicrotypeSet[protrusion]{basicmath} % disable protrusion for tt fonts
}{}
\usepackage[margin=1in]{geometry}
\usepackage{hyperref}
\hypersetup{unicode=true,
            pdftitle={Zajęcia z R},
            pdfborder={0 0 0},
            breaklinks=true}
\urlstyle{same}  % don't use monospace font for urls
\usepackage{graphicx,grffile}
\makeatletter
\def\maxwidth{\ifdim\Gin@nat@width>\linewidth\linewidth\else\Gin@nat@width\fi}
\def\maxheight{\ifdim\Gin@nat@height>\textheight\textheight\else\Gin@nat@height\fi}
\makeatother
% Scale images if necessary, so that they will not overflow the page
% margins by default, and it is still possible to overwrite the defaults
% using explicit options in \includegraphics[width, height, ...]{}
\setkeys{Gin}{width=\maxwidth,height=\maxheight,keepaspectratio}
\IfFileExists{parskip.sty}{%
\usepackage{parskip}
}{% else
\setlength{\parindent}{0pt}
\setlength{\parskip}{6pt plus 2pt minus 1pt}
}
\setlength{\emergencystretch}{3em}  % prevent overfull lines
\providecommand{\tightlist}{%
  \setlength{\itemsep}{0pt}\setlength{\parskip}{0pt}}
\setcounter{secnumdepth}{0}
% Redefines (sub)paragraphs to behave more like sections
\ifx\paragraph\undefined\else
\let\oldparagraph\paragraph
\renewcommand{\paragraph}[1]{\oldparagraph{#1}\mbox{}}
\fi
\ifx\subparagraph\undefined\else
\let\oldsubparagraph\subparagraph
\renewcommand{\subparagraph}[1]{\oldsubparagraph{#1}\mbox{}}
\fi

%%% Use protect on footnotes to avoid problems with footnotes in titles
\let\rmarkdownfootnote\footnote%
\def\footnote{\protect\rmarkdownfootnote}

%%% Change title format to be more compact
\usepackage{titling}

% Create subtitle command for use in maketitle
\providecommand{\subtitle}[1]{
  \posttitle{
    \begin{center}\large#1\end{center}
    }
}

\setlength{\droptitle}{-2em}

  \title{Zajęcia z R}
    \pretitle{\vspace{\droptitle}\centering\huge}
  \posttitle{\par}
    \author{}
    \preauthor{}\postauthor{}
    \date{}
    \predate{}\postdate{}
  

\begin{document}
\maketitle

\subsection{Wprowadzenie}\label{wprowadzenie}

\textbf{R} -- język

\textbf{Rstudio} -- program do języka R

RStudio składa się z kilku podokien i narzędzi:

\begin{itemize}
\item
  Konsola (\emph{Console}) -- tutaj możesz wpisywać bezpośrednio kod,
\item
  Okno środowiska (\emph{Environment}) -- wyświetlane tu są wszystkie
  zapisane w pamięci zmienne, jak i funkcje. Środowisko można zapisać,
  jak również wczytać. Używając tego okna można również importować dane
  z zewnątrz oraz przejrzeć historię wpisywanych linii kodu.
\item
  Okno z zakładkami -- tutaj możemy przeglądać strukturę plików na dysku
  (\emph{Files}), wyświetlać wykresy (\emph{Plots}), przeglądać
  zainstalowane pakietu (\emph{Packages}), szukać pomocy na temat
  funkcji z pakietów (\emph{Help}).
\item
  W R najczęściej chcemy pisać skrypty składające się z wielu linii,
  które następnie będziemy wykonywać. Do otwarcia nowego skryptu służy
  ikona Add R Script lub kombinacja klawiszy Ctrl+Shift+N.
\end{itemize}

Funkcje -- zazwyczaj skonstruowane w sposób:

nazwa\_funkcji(x, y, \ldots{})

\begin{itemize}
\tightlist
\item
  x -- dane wejściowe
\item
  y -- dalsze argumenty, które mogą być Na przykład w funkcji
  \href{https://stat.ethz.ch/R-manual/R-devel/library/utils/html/read.table.html}{\texttt{read.csv()}},
  która służy do wczytywania danych w formacie csv:
\end{itemize}

\texttt{read.csv(file,\ header\ =\ TRUE,\ sep\ =\ ",",\ quote\ =\ "\textbackslash{}"",\ dec\ =\ ".",\ fill\ =\ TRUE,\ comment.char\ =\ "",\ ...)}

Pakiety -- instalacja i wczytywanie: install.packages() library() Help
Typy obiektów

Wczytywanie bazy danych: Najczęściej w formacie .csv \textgreater{} data
frame Poszczególne kolumny \textgreater{} \$


\end{document}
